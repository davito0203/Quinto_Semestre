%%-------------------------------------------------------------------
%% Definici�n del tipo de documento
%%-------------------------------------------------------------------
\documentclass[twocolumn,10pt]{C:/texmf/tex/IEEEtran}  %para usar la libreria en espa�ol activa esta l�nea y desactiva la siguiente
%\documentclass[twocolumn,jorunal]{IEEEtran}            %para usar la libreria en ingles activa esta l�nea y desactiva la anterior
%%-------------------------------------------------------------------
%% Paquetes y librer�as que se van a cargar
%%-------------------------------------------------------------------
\usepackage[spanish]{babel,varioref}
\selectlanguage{spanish}
\usepackage[ansinew]{inputenc}
\usepackage{graphics,graphicx,color,colortbl}
\usepackage{times}
\usepackage{subfigure}
\usepackage{wrapfig}
\usepackage{multicol}
\usepackage{cite}
\usepackage{url}
\usepackage[tbtags]{amsmath}
\usepackage{amsmath,amssymb,amsfonts,amsbsy}
\usepackage{bm}
\usepackage{algorithm}
\usepackage{algorithmic}
\usepackage[all]{xy}
\usepackage[centerlast, small]{caption}
\usepackage[colorlinks=true, citecolor=blue, linkcolor=blue, urlcolor=blue, breaklinks=true]{hyperref}%\usepackage[dvips, pdfstartview=FitH, bookmarks=true,hypertexnames=false, letterpaper,linktocpage,colorlinks=true, citecolor=blue, linkcolor=blue, urlcolor=blue, breaklinks=true]{hyperref}
%%-------------------------------------------------------------------
%% Inicio del documento
%%-------------------------------------------------------------------
\begin{document}
%%-------------------------------------------------------------------
%% T�tulo
%%-------------------------------------------------------------------
\title{Compuertas TTL y CMOS}
%%-------------------------------------------------------------------
%% Autor y otros datos
%%-------------------------------------------------------------------
\author{David Ricardo Mart�nez Hern�ndez \\ C�digo: 261931}
\maketitle

\markboth{Universidad Nacional de Colombia}{}

%%-------------------------------------------------------------------
%% Otras definciones en espa�ol
%%-------------------------------------------------------------------
\floatname{algorithm}{Algoritmo}
%%---------------------------------------------------------
%% Resumen y Palabras clave
%%--------------------------------------------------------
\begin{abstract}
.
\end{abstract}

\begin{keywords}
ALTO, AND, BAJO, CMOS, Compuerta, NOT, operador l�gico, OR, tabla de verdad, TTL, XOR.
\end{keywords}
%%---------------------------------------------------------
%% Cuerpo del trabajo
%%---------------------------------------------------------
\section{Introducci�n}
\noindent
La l�gica bipolar
\section{Materiales y M�todos}
\noindent
Para desarrollar este laboratorio fue necesaria la utilizaci�n de:

\begin{itemize}
	\item Compuertas AND, NOT, OR, XOR.
	\item
\end{itemize}


\section{Desarrollo te�rico de los circuitos implementados}
\noindent

\subsection{Simulaciones}
\noindent

\subsection{An�lisis}
\noindent

\subsection{Resultados}
\noindent

\section{Conclusiones}
\begin{itemize}
	\item .
\end{itemize}
\begin{itemize}
	\item .
\end{itemize}
%%---------------------------------------------------------
%% Bibliograf�a
%%---------------------------------------------------------
\bibliographystyle{ieeetran}
\begin{thebibliography}{1}

\bibitem{La86} Dorf Svoboda.
{\em "`Circuitos El�ctricos"'}.
Alfaomega, 2006.
\bibitem{La86} C. J. Savant.
{\em "`Dise�os Electr�nicos: Circuitos de Sistema"'}.
Prentice-Hall, 2006.

\end{thebibliography}
%%---------------------------------------------------------
%% Fin del documento
%%---------------------------------------------------------
\end{document}